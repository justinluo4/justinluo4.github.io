% Document class and font size
\documentclass[letterpaper,10pt]{extarticle}

% Packages
\usepackage[utf8]{inputenc} % For input encoding
\usepackage{geometry} % For page margins
\geometry{letterpaper, margin=0.4in} % Set paper size and margins
\usepackage{titlesec} % For section title formatting
\usepackage{enumitem} % For itemized list formatting
\usepackage{hyperref} % For hyperlinks
\usepackage{setspace}
\setstretch{1.01}
% Formatting
\setlist{noitemsep} % Removes item separation
\titleformat{\section}{\large\bfseries}{\thesection}{1em}{}[\titlerule] % Section title format
\titlespacing*{\section}{0pt}{\baselineskip}{\baselineskip} % Section title spacing

% Begin document
\begin{document}

% Disable page numbers
\pagestyle{empty}

% Header
\begin{center}
\textbf{\Large JUSTIN LUO}\\[3pt] % Name
\href{mailto:jluo2@caltech.edu}{jluo2@caltech.edu} | (858) 793-1618 | \href{https://www.linkedin.com/in/justin-luo-471a48219}{linkedin.com/in/justin-luo-471a48219} | US Citizen | \href{https://github.com/justinluo4}{github.com/justinluo4}% Contact info
\end{center}
\begin{center}
Highly motivated Computer Science major at Caltech with a strong foundation in mathematics and engineering. Actively seeking opportunities to contribute to technology development with real-world applications.
\end{center}
% Education Section
\section*{EDUCATION}
\noindent
\textbf{California Institute of Technology (Caltech)}, Pasadena, CA \hfill September 2023 | Present\\ % University name and location
Major in Computer Science w/ Minor in Robotics \hfill GPA 4.04 out of 4.0


\section*{SKILLS}
\begin{itemize}
    \item \textbf{Platforms:} Python, C++, C, Java, Javascript, Node.js, SQL, R, CUDA, OCaml, GLSL, HTML5, Matlab, Git, PyTorch, TensorFlow, MuJoCo, Blender, ROS, OpenGL, Kubernetes, Docker, Linux \& Windows Sysadmin
    
    \item \textbf{Concepts:} Learning Systems, Data Mining, Deep Learning, Large Language Models, Software Design, Algorithms, Computer Systems, Robotics, Reinforcement Learning, Computer Vision, Graphics, Parallel Computing, Real Time Simulation
\end{itemize}



% % Additional Experience or Volunteer Work
% \section*{PROJECTS}
% \noindent
% \textbf{Robotic Camera Tracking System} \hfill Pasadena, CA\\ % Project or organization name and location
% \textit{Project Lead} \hfill September 2023 | December 2023\\ % Position and duration
% Designed and created a fully autonomous camera + arm system capable of following multiple objects in real time. Implemented computer vision using OpenCV in Python. Project was selected by Dr. Gunter Niemeyer as an example for future labs.
% % Projects Section

\section*{AWARDS \& ACCOLADES}
\begin{itemize}
    \item Putnam Top 300, 4x AIME Qualifier, Harvard-MIT Mathematics Tournament 20th place overall 
    \item IMC Prosperity 12th Place
    % , 2021 \& 2022 Socal Cyber Cup 1st Place Team
    \item USA Computing Olympiad Gold, Cyberpatriot Cybersecurity National Finalist
    \item USA Physics Olympiad Semifinalist
    \item Eagle Scout 
\end{itemize}
% End document


\section*{WORK EXPERIENCE}
\noindent
\textbf{Teaching Assistant} \hfill California Institute of Technology\\ % Project name and location
\textbf{\textit{ME/CS/EE 129 - Experimental Robotics}} \hfill \textit{Spring 2025} \\
Guiding small groups in creating an automated exploration robot, integrating sensors and implementing interrupt-driven and multi-threaded architectures in the graduate level course.\\
\textbf{\textit{CS 12 - Introduction to Prototyping}} \hfill \textit{Winter 2025} \\% Project link and duration
 Assisted 50+ students in designing and creating an open-ended project, providing a foundational experience in prototyping. 
 % Projects ranged between robotic arms, autonomous drones, and much more. 
 \vspace{1mm} \\
\textbf{\textit{ME 8 - Introduction to Robotics}} \hfill \textit{Fall 2024} \\% Project link and duration
 Led 40+ students in designing a fully autonomous camera \& arm system, requiring teaching proficiency in Python and CAD. Responsible for guiding several teams in achieving success in the project-based course.
 

%  \noindent 
% \textbf{Positron Scientific} \hfill San Diego, CA\\ % Company name and location
% \textit{Software Engineer Intern} \hfill \textit{Summer 2023}\\ % Position and duration
% \noindent 
% Developed a gamma camera detector data analysis program, allowing users to analyze and visualize latent noise within detector QC scans. Implemented an automated system to test the detector’s fit with the design document specifications. % Job responsibilities and achievements


\section*{RESEARCH EXPERIENCE}
\textbf{Research Internship w/ UCSD Su Lab} \hfill San Diego, CA\\ % Project name and location
\textit{Research Intern} \hfill \textit{Summer 2025} \\% Project link and duration
Engineering algorithms for task-aware mesh decomposition and grasp detection on objects for a RL manipulation environment. Contributing to Maniskill, a comprehensive RL manipulation training library developed by the Su Lab.  \\\\
% Engineered data pipelines for rendering \& animating arbitrary animal models using RL, then fine-tuning state-of-the-art point tracking models via cloud compute (VAST).
\noindent
\textbf{Undergraduate Research w/ Caltech Perona Lab} \hfill Pasadena, CA\\ % Project name and location
\textit{Undergraduate Researcher} \hfill \textit{Summer 2024} \\% Project link and duration
Researched the novel use of Reinforcement Learning (RL) to generate synthetic datasets for point tracking in the Perona Vision Lab. In-house research grant awarded by the Caltech SURF Fellowship. Presented at CVPR CV4Animals 2025. \\\\
% Engineered data pipelines for rendering \& animating arbitrary animal models using RL, then fine-tuning state-of-the-art point tracking models via cloud compute (VAST).
\noindent
\noindent
\textbf{Research Project w/ Dr. Makoto Miyakoshi @ UCSD} \hfill San Diego, CA\\ % Project name and location
\textit{Research Assistant} \hfill \textit{September 2021 | April 2023} \\% Project link and duration
Researched the reliability, durability, and performance of ICA, an advanced algorithm used for EEG processing and other signal analysis applications. Published into Frontiers in Computational Neuroscience: \href{https://doi.org/10.3389/frsip.2023.1064138}{https://doi.org/10.3389/frsip.2023.1064138}.
% \noindent
% \textbf{Pioneer Research} \hfill June 2022 | September 2022 \\% Project link and duration
% Researched the feasibility and robustness of the Quantum Fourier Transform in real applications using noisy simulation techniques. Worked with Dr. Anthony Hoffman of the University of Notre Dame to explore new areas of quantum computing. 
% Experience Section






\section*{ACTIVITIES \& ORGANIZATIONS}
\noindent \textbf{Caltech Quantitative Finance Club} \hfill September 2023 | Ongoing

\noindent \textbf{Caltech Robotic Manipulation Club} \hfill September 2024 | Ongoing

\noindent \textbf{Caltech Orchestra \& Wind Orchestra} / \textit{Percussionist} \hfill September 2023 | Ongoing

% \noindent \textbf{Boy Scouts w/ Troop 765} / \textit{Eagle Scout, Senior Patrol Leader} \hfill 2015 | 2023

%\noindent \textbf{CCA Math Team} / \textit{President} \hfill August 2019 | June 2023


\end{document}


